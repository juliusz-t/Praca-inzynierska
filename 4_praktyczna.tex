\chapter{Implementacja sterowania NAO}
\section{Interfejs robota Nao}

\subsubsection{NAOqi} System operacyjny autorstwa firmy Aldebaran, pozwalający na sterowanie robotem NAO, programowanie go oraz utworzenie wirtualnego agenta podłączonego do symulacji. Producent dostarcza zestaw narzędzi, pozwalających na pisanie programów w językach C++, Java lub Python. NAOqi może zostać uruchomione niezależnie od innych aplikacji z poziomu terminala.

\subsubsection{Choreographe} Aplikacja okienkowa, umożliwiająca tworzenie scenariuszy działania robota w formie graficznej. Choreographe dostarcza bibliotekę akcji w postaci bloków, możliwych do połączenia przez programistę w automat skończony. Dzięki aplikacji można w łatwy sposób podłączyć się do fizycznego robota, utworzyć symulację, bądź połączyć się z działającą w systemie instacją NAOqi. W jednej z zakładek można zobaczyć wizualizację stanu robota, która działa wydajniej, niż inne dostępne rozwiązania.

\subsubsection{ROS} Oprogramowanie ułatwiające sterowanie robotami. Zapewnia zbiór narzędzi umożliwiających 

tworzenie symulacji
nadawanie i odbieranie wiadomości, szkielet architektury wiadomości i serwisów
zewnętrzne biblioteki, integracja z Pythonem, C++, przeglądarkami internetowymi
Oprogramowanie ułożone w pakiety, ROS potrafi sam zadbać o odpowiednie zależności.

Integracja z ROSem może otworzyć dodatkowe możliwości, takie jak komunikacja z otoczeniem i innymi robotami, poszerzenie zmysłów, poprzez integrację zewnęrznych czujników 



\section{SMACH}

\subsection{Koncepcja zastosowania}
Robot Nao nie posiada wbudowanego automatu skończonego. Dlatego celowe wydaje się zintegrowanie jego działania z zewnętrznym narzędziem do tworzenia automatów. 
Zadanie można zrealizowac z użyciem SMACH (skrót od "State Machine"). Biblioteka ta została stworzona przez deweloperów ROSa, ale jest od niego niezależna i może być używana osobno. 

Skończone maszyny stanów są wykorzystywane w robotyce od 

\subsection{Implementacja}
Pakiet SMACH jest instalowany domyślnie z ROSem. Z poziomu Pythona biblioteka jest importowana bez żadnych dodatkowych działań.

\subsection{Weryfikacja koncepcji}
SMACH pozwala na tworzenie wysokopoziomowej warstwy sterowania robotem. Możliwe jest tworzenie prostych scenariuszy w postaci maszyn stanu. Dzięki modułowości architektury SMACHa, można wykorzystać je jako elementy większych sceraniuszy. 

Dzięki integracji z ROSem SMACH udostępnia interfejs do analizy stanu automatu w czasie rzeczywistym. 

Działanie automatu skończonego ciężko nazwać inteligentnym. Robot nie planuje samodzielnie zadań. Programista jest zmuszony do przewidzenia wszystkich możliwych scenariuszy działania. 

SMACH wymaga zdefiniowania wszystkich możliwych stanów i zasad przejścia między stanami przed rozpoczęciem swojego działania. Jego działanie ma charakter statyczny i nie może być dostosowywane w trakcie realizacji programu. Wymaga to od programisty mądrego przemyślenia architektury systemu i uwzględnienia ewentualnych problemów przy realizacji zadania (np. kolizji robota z przeszkodą). 

Mimo to, właśnie automaty skończone są podstawową metodą wykorzystywaną do interakcji robot - człowiek.

\section{RGOAP}

\subsection{Koncepcja zastosowania}
Maszyny stanu pozwalają na realizację przez robota scenariuszów, jednak każdy z nich musi być zaplanowany osobno. Wymaga to dużego nakładu pracy, więc zasadnym jest podjęcie próby automatyzacji tego procesu. GOAP na podstawie warunków i efektów zdefiniowanych wcześniej akcji pozwala na automatyczne łączenie ich w proste lub bardziej skomplikowane scenariusze. 

\subsection{Implementacja}
RGOAP powstało w 2013 roku, jako pakiet ROSa w wersji Groovy. Do budowy pakietu należy wykorzystać narzędzie rosbuild. Jest to problematyczne, ponieważ aktualnie wspieranym narzędziem tworzenia i budowy pakietów jest Catkin. Z tego powodu w trakcie realizacji pracy przebudowano strukturę plików pakietu, catkinizując RGOAP. Efekt pracy można pobrać z repozytorium autora: \url{https://github.com/juliusz-t/executive_rgoap/tree/catkin}

\subsection{Weryfikacja koncepcji}

Projektowanie scenariuszy w FSM i GOAP może mieć bardzo podobny efekt. Jednak w FSM naturalniejszym rozwiązaniem wydaje się okreslenie wierzchołków grafu jako statycznych stanów i krawędzi grafu, jako dynamicznych akcji. GOAP będąc bardziej intuicyjny zmusza do okreslania wierzchołków grafu jako akcji, a krawędzi jako momentów w których stan świata jest czasowo statyczny. Uzyskane w ten sposób grafy mogą być dualne względem siebie.

\subsection{Możliwość zastosowania w poznawczych modelach umysłu}

% \section{Porównanie SMACH i RGOAP}
% FSM jako statyczny, GOAP jako dynamiczny system podejmowania decyzji.
