\chapter{Problem sterowania robotem społecznym}
% TODO:
% Definicja agenta mobilnego. Obiekt sa model. Warstwowość agentów.
% Model BDI, intuicyjność rozwiązania, 
% Robot NAO, możliwości sterowania, choreograph, naoqi, ROS
% Brak wysokopoziomowej warstwy sterowania

\section{Agent mobilny}
Agent stanowi formę pośrednią między programem komputerowym a robotem. Najłatwiej jest opisać agenta jako obiekt (w rozumieniu programowania obiektowego). Obiekty składają się pól i posiadają metody. Dostęp do zawartości pól i do wywoływania metod może być ograniczany (prywatny) lub ogólnie dostępny (publiczny), ale decyzja o podjęciu przez obiekt jakiegoś działania delegowana jest do wyższych warstw programu, np. nadrzędnej funkcji, lub programisty wywołującego metodę z poziomu linii komend. 

Różnicą między obiektami a agentami jest autonomiczność tych drugich. Agent na podstawie swojego stanu lub wiedzy o świecie może sam zdecydować o podjęciu działania. Ze względu na Wyróżnia się agentów czysto reaktywnych i agentów z parametrem wewnętrznym, którzy dodatkowo przechowują informację o kolejnych stanach świata i są w stanie modelować konsekwencje swoich działań. Z formalnego punktu widzenia agent czysto reaktywny musi posiadać przynajmniej dwie metody – see() pozwalającą na pobranie wiedzy o świecie i action() pozwalającą na oddziaływanie na świat. Agent z parametrem wewnętrzym posiada dodatkową metodę next() tworzącą model świata na podstawie infomacji zwróconych przez see(). Kolejne modele świata są zapamiętywane w bazie danych.  

Działanie agentów z parametrem wewnętrznym może być traktowane jako podróż po grafie skierowanym, którego wierzchołkami są różne modele świata, a krawędziami dostępne akcje. Planowanie działań sprowadza się do wyznaczenia ścieżki między początkowym a końcowym stanem świata.

Robot stanowi interfejs pomiędzy agentem a światem rzeczywistym.  (agent upostaciowiony robot)

\section{Model BDI}
Model belief-desire–intention (model przekonań-pragnień-intencji) to metodyka projektowania agentów. 

Naturalnym dla człowieka jest zaspokajanie swoich potrzeb poprzez określenie celów do realizacji, a następnie 

Dekompozycja porządanego stanu na działania, które do niego prowadzą.

% Planner, 

% model symuluje zachowanie człowieka

\section{Robot społeczny NAO}
% Nao posiada dobrze rozwinięte niższe warstwy sterowania, nie ma jednak narzędzi do 