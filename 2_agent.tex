\chapter{Problem sterowania robotem społecznym}

\section{Agent mobilny}
Słownik języka polskiego nie definiuje słowa agent w takim rozumieniu, w jakim będzie ono używane w poniższej pracy, ponieważ jest to zapożyczenie z języków obcych. Jedna z definicji oksfordzkiego słownika języka angielskiego mówi: "A person or thing that takes an active role or produces a specified effect", co można przetłumaczyć jako "Osoba lub rzecz, która bierze udział lub przyczynia się do realizacji określonego efektu". 

Pojęcie agenta pochodzi z informatyki, gdzie oznacza autonomiczny program realizujący zadanie, np. zbieranie i przetwarzanie infomacji.

Najłatwiej jest opisać agenta jako obiekt (w rozumieniu programowania obiektowego) [1]. Obiekty składają się pól i posiadają metody. Dostęp do zawartości pól i do wywoływania metod może być ograniczany (prywatny) lub ogólnie dostępny (publiczny), ale decyzja o podjęciu przez obiekt jakiegoś działania delegowana jest do wyższych warstw programu, np. nadrzędnej funkcji, lub programisty wywołującego metodę z poziomu linii komend. 

Cechą wyróżniającą agentów od pozostałych obiektów jest autonomiczność działania. Agent na podstawie swojego stanu lub wiedzy o świecie może samodzielnie zdecydować o podjęciu działania. Pojęcie agenta przeszło z informatyki do robotyki, gdzie jest rozumiane jako obiekt przekształcający percepcję na inteligentne działanie. Robot jest agentem uprzedmiotowionym, czyli posiadającym fizyczną formę i oddziałującym z realnym światem.

\subsection{Klasyfikacja CERT}

%\begin{wraptable}{p}{0.5\linewidth}
%\caption{Typy agentów}
%\label{tab:cert}
%\begin{tabular}{ | l | p{6cm} | } \hline
%C & zbyt prymitywny do realizacji jakiegokolwiek zadania \\ \hline
%CE  & mogący jedynie oddziaływać na otoczenie \\ \hline
%CR & jedynie gromadzący dane \\ \hline
%CT & zdolny do komunikacji, mogący przetwarzać dane, "serwer", "obliczenia w chmurze" \\ \hline
%CER & agent niezależny \\ \hline
%CET & zdalny efektor  \\ \hline
%CRT & zdalny czujnik \\ \hline
%CERT & kompletny agent \\ 
%\hline
%\end{tabular}
%\end{wraptable} 

W pracy [2] prof. Cezary Zieliński przedstawia osiem typów agentów, podzielonych ze względu na komponenty wchodzące w ich skład. Możemy wyróżnić cztery podsystemy: sterowania (C), stanowiący podstawowy element struktury agenta; efektorów (E), oddziaływujący na otoczenie; receptorów (R), zbierający informacje o stanie otoczenia; oraz komunikacji (T), pozwalający na wymianę informacji z innymi agentami. Podsystem sterowania jest niezbędny, podczas gdy pozostałe są opcjonalne. Pozwala to utworzyć klasyfikację, ze względu na posiadane podsystemy. Typy agentów opisano w tabeli 2.1. % \ref{tab:cert}. 

W środowisku wieloagentowym podsystem komunikacji pełni kluczową rolę. Robot społeczny musi być zdolny do tworzenia społeczności, a więc nadawać i odbierać informacje. Struktura podsystemów agenta może być stała, lub zmienna. Niedoścignionym wzórem dla projektantów robotów są organizmy żywe charakteryzujące się zdolnością do: adaptacji do zmiennych warunków środowiska i poszerzania swoich możliwości dzięki wymianie informacji. 

\subsection{Klasyfikacja ze względu na podsystem sterowania}
% dr inż. Michał Gnatowski

Ze względu na sposób działania podsystemu C można wyróżnić \textbf{agentów czysto reaktywnych}, którzy reagują na aktualny stan świata, oraz \textbf{agentów z parametrem wewnętrzym}, analizujących również poprzednie stany świata. 

\subsubsection{Agent czysto reaktywny}

Działanie podsystemów R oraz E można zamodelować przy pomocy dwóch funkcji: see() zwracającej stan świata, oraz action() oddziałującej na świat. Zadaniem podsystemu sterowania jest przetworzyć wyjście funkcji see() na wejście funkcji action(). Opisywany agent zrealizuje zadanie bezpośrednio, przetwarzając dane ad hoc.

\subsubsection{Agent z parametrem wewnętrzym}

Agent realizuje zadanie na podstawie wygenerowanych modeli świata. Modele te mogą być przechować w bazie danych. Oprócz podsystemów E i R modelowany jest również C z użyciem funkcji \textit{next()} którą wywołana jest na dwa sposoby. Argumentem tej funkcji mogą być informacje z otoczenia (wyjście funkcji see()). Agent tworzy wtedy aktualny model świata. Argumentem może być również para: istniejący już model i jedna z dostępnych akcji. Pozwala to na przewidywanie konsekwencji akcji i tworzenie planów. 

Działanie agentów z parametrem wewnętrznym może być traktowane jako podróż po grafie skierowanym, którego wierzchołkami są różne modele świata, a krawędziami dostępne akcje. Planowanie działań sprowadza się do wyznaczenia ścieżki między początkowym a końcowym stanem świata.

\section{Model BDI}
Model belief-desire–intention (model przekonań-pragnień-intencji) to metodyka projektowania agentów. 

\subsubsection{Przekonania}
Opisują aktualny stan świata postrzegany przez agenta. Używa się słowa "przekonania", a nie "wiedza" dla pokreślenia, że agent może wierzyć w zdania, które niekoniecznie są prawdziwe. Przekoniania zebrane są w zbiór zdań. Do tego zbioru mogą należeć również reguły wnioskowania. Dzięki nim na podstawie istniejących przekonań można wyprowadzić inne. Z tego powodu model BDI musi zabierać system dowodzenia/automatycznego wnioskowania.

\subsubsection{Pragnienia}
Opisują stan świata porządany przez agenta. 
Pragnienia urzeczywistniają się poprzez stawianie sobie celów. Pragnienie może być zdefiniowane ogólnie, natomiast cel musi określać warunki sukcesu. Pragnienia mogą być sprzeczne, ale agent nie może jednocześnie realizować dwóch sprzecznych celów.

\subsubsection{Intencje}
Opisują hierarhię pragnień agenta. 



% Naturalnym dla człowieka jest zaspokajanie swoich potrzeb poprzez określenie celów do realizacji, a następnie 

% Dekompozycja porządanego stanu na działania, które do niego prowadzą.

% Planner, 
% model symuluje zachowanie człowieka

\section{Robot społeczny}
Robot społeczny, to uprzedmiotowiony agent, który posiada . Robot stanowi interfejs pomiędzy agentem a światem rzeczywistym. 
Działa deliberatywnie – cyklicznie wykonuje trzy działania: odczuwa, planuje i działa.
% W pracy Fonga, Nourbakhsha i Dautenhahna wymieniono następujące
\subsubsection{Cechy robota społecznego}
\begin{itemize}
    \setlength\itemsep{0em}
    \item wyrażanie i postrzeganie emocji;
    \item zdolność do wysokopoziomowej komunikacji werbalnej; 
    \item wykorzystywanie naturalnych gestów, "mowy ciała", do komunikacji niewerbalnej;
    \item rozpoznawanie i zapamiętywanie innych agentów, lub ludzi;
    \item nawiązywanie i utrzymywanie relacji społecznych;
    \item posiadanie własnego charakteru i cech osobowości;
    \item zdolność do nauki i rozwiniania umiejętności społecznych.
\end{itemize}

%  (agent upostaciowiony robot - embodied agents)

\section{Robot NAO}
% Nao posiada dobrze rozwinięte niższe warstwy sterowania, nie ma jednak narzędzi do 